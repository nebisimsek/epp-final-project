\documentclass[12pt, a4paper]{article}

\usepackage{float}
\usepackage{booktabs, tabularx}
\usepackage{multirow,array}
\usepackage[utf8]{inputenc}
\usepackage{blindtext}
\usepackage{hyperref}
\usepackage{natbib}
\bibliographystyle{apa-good}
\usepackage[nottoc]{tocbibind}
\usepackage{amssymb}
\usepackage{tcolorbox}
\usepackage{caption}
\usepackage{amsthm}
\usepackage{amsmath}
\usepackage{graphicx}
\usepackage{subfigure}
\usepackage{marginnote} %notes for authors
\usepackage{imakeidx}
\usepackage{cleveref}
\usepackage{threeparttable}
\usepackage{subcaption}
\usepackage{fullpage}
\usepackage{setspace}\onehalfspacing
\setstretch{1.5}
\usepackage{abstract}
%try idea: produce a header with our name in each section

%Acronyms%
\usepackage{acronym}
\acrodef{cdf}[cdf]{\textbf{Cumulative Distribution Function}}
\acrodef{pdf}[pdf]{\textbf{Probability Density Function}}
\acrodef{pmf}[pmf]{\textbf{Probability Mass Function}}
\acrodef{iid}[i.i.d.]{\textbf{independent and identically distributed}}
\acrodef{did}[DiD]{\textbf{Difference-in-Difference}}
\acrodef{ols}[OLS]{\textbf{Ordinary Least Squares}}
\acrodef{twfe}[TWFE]{\textbf{Two-Way Fixed Effects}}
\acrodef{twfer}[TWFEr]{\textbf{Two-Way Fixed Effects Regression}}
\acrodef{OWFE}[OWFE]{\textbf{One-Way Fixed Effects}}
\acrodef{mcs}[MCS]{\textbf{Monte Carlo Simulation}}





\begin{document}

\title{Comparison of Three Panel Data Estimators}
\author{\textbf{\LARGE{Nebi Simsek}}}
\date{}

\maketitle
\vspace*{3em}
\tableofcontents
\pagebreak
% \pagenumbering{Roman}

% \include{./title.tex}
% \tableofcontents
% \clearpage
% \listoffigures
% \clearpage
% \listoftables
% \clearpage
% \pagenumbering{arabic}

\section{Monte Carlo Simulations}
\addcontentsline{toc}{section}{Monte Carlo Simulation}
In the previous section, we introduce \ac{OWFE} and \ac{twfe} estimators.
These estimators are especially needed when the fixed errors are correlated with the covariates.
Correlation with the error terms results in bias that may lead to wrong interpretations.
Luckily, it is possible to eliminate the bias term with the transformations mentioned in the previous
section if the model satisfies certain assumptions. In this section, we conducted some
Monte Carlo Simulations to see how \ac{OWFE} and \ac{twfe} estimators
react to certain structural changes in the model.
\subsection{Data Generating Process}
\addcontentsline{toc}{subsection}{Data Generating Process}
For simulations, we used a multivariate case, and the dependent variable is
explained by five different characteristics. For each unit at a specific time,
the covariates are structured in three parts: Unit-specific randomly
generated variables, time-specific randomly generated variables
and a variation over time.
$$X_{it}^{unit} \sim N(0, \Sigma_{X}); \quad
X_{it}^{time} \sim N(0, \Sigma_{X}); \quad
X_{it}^{var} \sim N(0, \Sigma_{X})
$$
For all three structures, we used the same randomly generated variance-covariance
matrix for simplicity. $N$ represents the number of units, and $T$ represents
the number of  periods. $X_{it}^{unit}$ is the same value for the unit $i$
at each time $t = 1 \ldots T$ while $X_{it}^{time}$ is the same value for
all the units $i = 1 \ldots N$ at time $t$. Using the third structure $X_{it}^{var}$,
we generated the variation for each unit across time. We sum up these three terms
and obtain the covariates for unit $i$ at time $t$.
$$X_{it} = X_{it}^{unit} + X_{it}^{time} + c_{var} X_{it}^{var}$$
Here it is easy to see that we can control for the variation across time with the constant $c_{var}$. Setting this value to $0$, we will get unidentified TWFE estimates.
Depending on the number of characteristics, we may also get an unidentified \ac{OWFE} results since we may not satisfy the full rank assumption. Following, we generate the error terms and trend variables.
$$\epsilon_{it} \sim N(0, 1); \quad w_{i} \sim N(2, 0.5)$$
$$u_i = c_{unit}X_{(2)}^{unit} + \nu_{i}; \quad \nu_{i} \sim U(0,1);
\quad v_t = c_{time}X_{(2)}^{time}  + \xi_{t}; \quad \xi_{t} \sim U(0,1)$$

We generate time-invariant and unit-invariant errors correlated with the
second dimension of $X$ since we will be interested in estimates
for the second parameter. The $c_{unit}$ and $c_{time}$ constants determine
the correlation of error terms with this second dimension of $X$ covariates.
Assigning $0$ to these values will eliminate the correlation.
On the other hand, we can simply increase these constants to obtain higher correlations.
We introduced the time trend variable for each unit with $w_{i}$. The parameters for $w_{i}$ are
set arbitrarily to obtain positive trend variables. Finally, we set $\beta = [1, 5, 3, 3, 3]$
and obtain the $Y_{it}$.
$$Y_{it} = [1, 5, 3, 3, 3] X_{it} + v_{t} + c_{trend} t w_{i} + u_{i} + \epsilon_{it} $$
Here $t$ in front of $w_i$ represents the periods, and with another constant $c_{trend}$, we will control the intensity of the time trend. If we change this value to $0$, we can eliminate the trended variables. In all our simulations, we set $N = 100$, $T = 20$ and simulate the results $500$ times. In the next part, we will control the values of constants we introduced during DGP and show how \ac{OWFE} and
\ac{twfe} estimators react to these changes.

\subsection{Results}
\addcontentsline{toc}{subsection}{Results}
\begin{figure}[htbp]
    \caption{Density Graph for $ \beta_2$ Under Different Unit Invariant Error Correlations}
    %\centerline{\includegraphics{../simulations/plots/time_invariant_corr.png}}
    \subfigure[]{\includegraphics[width=0.5\textwidth]{../bld/python/figures/c_unit/0.png}}
    \subfigure[]{\includegraphics[width=0.5\textwidth]{../bld/python/figures/c_unit/0.05.png}} \\
    \subfigure[]{\includegraphics[width=0.5\textwidth]{../bld/python/figures/c_unit/0.15.png}}
    \subfigure[]{\includegraphics[width=0.5\textwidth]{../bld/python/figures/c_unit/0.4.png}}
    \caption*{\scriptsize{The figure plots the density graph of $\beta_2$
    obtained from different estimators. We have controlled the performance of
    estimators for different values of $c_{unit}$. We set $c_{time} = 0$, $c_{trend} = 0$
    and $c_{var} = 0.25$. We set $T = 20$, $N = 100$ and simulate the results
    for $500$ times. The resulting density graphs visualize the performance
    of three different estimators: Pooled OLS estimator, \ac{OWFE} estimator and  \ac{twfe} estimator.
    RMSE values for estimators can be found in \Cref{did_apendix} \Cref{table:rmse}(a).
    \label{fig:c1}
    }}
\end{figure}
It is important to understand the bias structure and justify not using Pooled OLS
estimator without transformation when we have correlated error terms. To see that,
we first look at the correlated time-invariant and set $c_{unit} > 0$ while
$c_{time} = 0$. In particular, we will use $c_{unit} \in \{0, 0.05, 0.15, 0.4\}$.
The resulting density graphs of estimates are shown in Figure \ref{fig:c1}.


Here we can see that as soon as we introduce the correlation with a positive
 $c_{unit}$, the pooled OLS estimates become unreliable. On the other hand,
 the performance of one-way and  \ac{twfe} estimators looks similar
 for all $c_{unit}$ values. Since we don't lose much with a  \ac{twfe}
 estimator, it looks like a safe choice for many analyses. However, when the
 variation over time is low, we may want to use \ac{OWFE} instead of  \ac{twfe}.
 To see that, we can control the variation of $X$ covariates over time with $c_{var}$ and
 observe the performance of the estimators. We will use $c_{var} \in \{0.03, 0.1, 0.25, 0.5\}$.
\begin{figure}[htbp]
    \caption{Density Graph for $\beta_2$ Under Low Variance Across Time}
    %\centerline{\includegraphics{../simulations/plots/var_over_time.png}}
    \subfigure[]{\includegraphics[width=0.5\textwidth]{../bld/python/figures/c_var/0.03.png}}
    \subfigure[]{\includegraphics[width=0.5\textwidth]{../bld/python/figures/c_var/0.1.png}} \\
    \subfigure[]{\includegraphics[width=0.5\textwidth]{../bld/python/figures/c_var/0.25.png}}
    \subfigure[]{\includegraphics[width=0.5\textwidth]{../bld/python/figures/c_var/0.5.png}}
    \caption*{\scriptsize{The figure plots the density graph of $\beta_2$
    obtained from different estimators. We have controlled the performance of
    estimators for different values of $c_{var}$. We set $c_{time} = 0$, $c_{trend} = 0$
    and $c_{unit} = 0.4$. We set $T = 20$, $N = 100$ and simulate the results
    for $500$ times. Graphs visualize the performance
    of three different estimators: Pooled OLS, \ac{OWFE} estimator and  \ac{twfe} estimator.
    RMSE values for estimators can be found in \Cref{did_apendix} \Cref{table:rmse}(b).
    \label{fig:c2}
    }}
\end{figure}

The \ac{OWFE} estimator is a much better estimator when the variation
across time is low since we eliminate the variation of covariates
further with the two-way within transformation.
However, after introducing sufficient variation, we observe that the \ac{twfe} estimator becomes even more
efficient than the \ac{OWFE} estimator since the decrease in the variation of residual terms dominates the decrease in the variation of X. Moreover, as we will see in Figure \ref{fig:c3},
introducing the trended variable on the right-hand side of the model will cause further efficiency loss for the \ac{OWFE} estimator.
\begin{figure}[t]
    \caption{Density Graph for $\beta_2$ With Trended Variables}
    %\centerline{\includegraphics{../simulations/plots/trend.png}}
    \subfigure[]{\includegraphics[width=0.5\textwidth]{../bld/python/figures/c_trend/0.03.png}}
    \subfigure[]{\includegraphics[width=0.5\textwidth]{../bld/python/figures/c_trend/0.1.png}} \\
    \subfigure[]{\includegraphics[width=0.5\textwidth]{../bld/python/figures/c_trend/0.25.png}}
    \subfigure[]{\includegraphics[width=0.5\textwidth]{../bld/python/figures/c_trend/0.5.png}}
    \caption*{\scriptsize{The figure plots the density graph of $\beta_2$
    obtained from different estimators. We have controlled the performance of
    estimators for different values of $c_{trend}$. We set $c_{time} = 0$, $c_{var} = 0.25$
    and $c_{unit} = 0.4$. We set $T = 20$, $N = 100$ and simulate the results
    for $500$ times. Graphs visualize the performance
    of three different estimators: Pooled OLS, \ac{OWFE} estimator and
    \ac{twfe} estimator. RMSE values for estimators can be found in \Cref{did_apendix} \Cref{table:rmse}(c).
    \label{fig:c3}
    }}
\end{figure}
Here we used $c_{trend} \in \{0.03, 0.1, 0.25, 0.5\}$.  As we increased the intensity of the trend, we saw that all estimators lost efficiency while
\ac{twfe} estimator is performing much better than the \ac{OWFE} estimator.
Moreover, as expected, introducing a correlated unit invariant error term will cause bias for the one-way fixed effect estimator.
To see that we will use $c_{time} \in \{0, 0.015, 0.05, 0.2\}$ and summarize the results in \Cref{fig:c4}.
\begin{figure}[t]
    \caption{Density Graph for $ \beta_2 $ Under Different Unit Invariant Correlations}
    %\centerline{\includegraphics{../simulations/plots/unit_invariant_corr.png}}
    \subfigure[]{\includegraphics[width=0.5\textwidth]{../bld/python/figures/c_time/0.png}}
    \subfigure[]{\includegraphics[width=0.5\textwidth]{../bld/python/figures/c_time/0.015.png}} \\
    \subfigure[]{\includegraphics[width=0.5\textwidth]{../bld/python/figures/c_time/0.05.png}}
    \subfigure[]{\includegraphics[width=0.5\textwidth]{../bld/python/figures/c_time/0.2.png}}
    \caption*{\scriptsize{The figure plots the density graph of $\beta_2$
    obtained from different estimators. We have controlled the performance of
    estimators for different values of $c_{time}$. We set $c_{trend} = 0$, $c_{var} = 0.25$
    and $c_{unit} = 0.4$. We set $T = 20$, $N = 100$ and simulate the results
    for $500$ times. Graphs visualize the performance
    of three different estimators: Pooled OLS, one-way fixed effect estimator and
    \ac{twfe} estimator. RMSE values for estimators can be found in \Cref{did_apendix} \Cref{table:rmse}(d).
    \label{fig:c4}}}
\end{figure}

In addition to our discussion about efficiency, the  \ac{twfe} estimator eliminates the bias when there are correlated unit-invariant error terms.
With that being said, of course, the dynamics of a real-world setting could be much more different and may not allow for a straightforward interpretation using the methods discussed here.
We may need more assumptions or control for other potential bias possibilities like reverse causality or selection bias. In this section, we tried to show some properties of Pooled, \ac{OWFE} and
\ac{twfe} estimators and discuss some cases where we would like to use one over the other. Proper interpretation of estimated values is critical to reaching causal inferences, and in our simulations, we hoped to give insights into such interpretations. Only using these methods with a proper discussion about the underlying model could lead us to sound causal interpretations.

In the next section, we will add a binary treatment variable to this model and introduce the generalized DiD model with a  specific focus on the underlying assumptions. Later with an application, we will give an example of how the methods discussed in this section can be used to reach the so-called sound causal interpretations.

\end{document}





% \documentclass[11pt, a4paper, leqno]{article}
% \usepackage{a4wide}
% \usepackage[T1]{fontenc}
% \usepackage[utf8]{inputenc}
% \usepackage{float, afterpage, rotating, graphicx}
% \usepackage{epstopdf}
% \usepackage{longtable, booktabs, tabularx}
% \usepackage{fancyvrb, moreverb, relsize}
% \usepackage{eurosym, calc}
% % \usepackage{chngcntr}
% \usepackage{amsmath, amssymb, amsfonts, amsthm, bm}
% \usepackage{caption}
% \usepackage{mdwlist}
% \usepackage{xfrac}
% \usepackage{setspace}
% \usepackage[dvipsnames]{xcolor}
% \usepackage{subcaption}
% \usepackage{minibox}
% % \usepackage{pdf14} % Enable for Manuscriptcentral -- can't handle pdf 1.5
% % \usepackage{endfloat} % Enable to move tables / figures to the end. Useful for some
% % submissions.

% \usepackage[
%     natbib=true,
%     bibencoding=inputenc,
%     bibstyle=authoryear-ibid,
%     citestyle=authoryear-comp,
%     maxcitenames=3,
%     maxbibnames=10,
%     useprefix=false,
%     sortcites=true,
%     backend=biber
% ]{biblatex}
% \AtBeginDocument{\toggletrue{blx@useprefix}}
% \AtBeginBibliography{\togglefalse{blx@useprefix}}
% \setlength{\bibitemsep}{1.5ex}
% \addbibresource{../../paper/refs.bib}

% \usepackage[unicode=true]{hyperref}
% \hypersetup{
%     colorlinks=true,
%     linkcolor=black,
%     anchorcolor=black,
%     citecolor=NavyBlue,
%     filecolor=black,
%     menucolor=black,
%     runcolor=black,
%     urlcolor=NavyBlue
% }


% \widowpenalty=10000
% \clubpenalty=10000

% \setlength{\parskip}{1ex}
% \setlength{\parindent}{0ex}
% \setstretch{1.5}


% \begin{document}

% \title{Comparison of Three Panel Data Estimators\thanks{Nebi Simsek, University of Bonn. Email: \href{mailto:s6nesims@uni-bonn.de}{\nolinkurl{s6nesims [at] uni-bonn [dot] de}}.}}

% \author{Nebi Simsek}

% \date{
%     {\bf Preliminary -- please do not quote}
%     \\[1ex]
%     \today
% }

% \maketitle


% \begin{abstract}
%     Some abstract here.
% \end{abstract}

% \clearpage


% \section{Introduction} % (fold)
% \label{sec:introduction}

% If you are using this template, please cite this item from the references:
% \citet{GaudeckerEconProjectTemplates}.

% The data set for the example project is taken from
% \url{https://www.stem.org.uk/resources/elibrary/resource/28452/large-datasets-stats4schools}.
% It contains data on smoking habits in the UK, with 1691 observations and 12 variables.
% We consider only 4 of the 12 features for the prediction of the variable
% \texttt{smoking}: \texttt{marital\_status}, \texttt{highest\_qualification},
% \texttt{gender} and \texttt{age}. We model the dependence using a Logistic model. All
% numerical features are included linearly, while categorical features are expanded into
% dummy variables. Figures below illustrate the model predictions over the lifetime. You
% will find one figure and one estimation summary table for each installed programming
% language.



% \begin{figure}[H]

%     \centering
%     \includegraphics[width=0.85\textwidth]{../python/figures/smoking_by_marital_status}

%     \caption{\emph{Python:} Model predictions of the smoking probability over the
%         lifetime. Each colored line represents a case where marital status is fixed to one
%         of the values present in the data set.}
%     \label{fig:python-predictions}

% \end{figure}


% \begin{table}[!h]
%     \input{../python/tables/estimation_results.tex}
%     \caption{\label{tab:python-summary}\emph{Python:} Estimation results of the
%         linear Logistic regression.}
% \end{table}




% % section introduction (end)



% \setstretch{1}
% \printbibliography
% \setstretch{1.5}


% % \appendix

% % The chngctr package is needed for the following lines.
% % \counterwithin{table}{section}
% % \counterwithin{figure}{section}

% \end{document}
